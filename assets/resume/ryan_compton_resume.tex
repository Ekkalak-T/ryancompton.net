\documentclass[margin,line]{res}

\oddsidemargin -.5in
\evensidemargin -.5in
\textwidth=6.0in
\itemsep=0in
\parsep=0in

\newenvironment{list1}{
  \begin{list}{\ding{113}}{%
      \setlength{\itemsep}{0in}
      \setlength{\parsep}{0in} \setlength{\parskip}{0in}
      \setlength{\topsep}{0in} \setlength{\partopsep}{0in} 
      \setlength{\leftmargin}{0.17in}}}{\end{list}}
\newenvironment{list2}{
  \begin{list}{$\bullet$}{%
      \setlength{\itemsep}{0in}
      \setlength{\parsep}{0in} \setlength{\parskip}{0in}
      \setlength{\topsep}{0in} \setlength{\partopsep}{0in} 
      \setlength{\leftmargin}{0.2in}}}{\end{list}}

\usepackage[colorlinks]{hyperref}


\begin{document}

\name{Ryan Compton - Resume \vspace*{.1in}}

\begin{resume}
\section{\sc Contact Information}
\vspace{.05in}
\begin{tabular}{@{}p{2in}p{4in}}
    ryan@ryancompton.net   &  \url{http://www.ryancompton.net} \\         
(310) 894-6829 & \\
\end{tabular}

\section{\sc Education}
    {\bf MS, PhD}, Mathematics, UCLA, 2006-2012\\
\vspace*{-.1in}
\begin{list1}
\item [] Dissertation: Sparsity promoting optimization in quantum mechanical signal processing.
\item [] Advisor: Chris Anderson
\end{list1}

    {\bf BA}, Mathematics/Physics, New College of Florida, 2002-2006\\
\vspace*{-.1in}
\begin{list1}
\item [] Thesis: Optimizing cover times with constraints.
\item [] Advisor: Patrick McDonald
\end{list1}

\section{\sc Employment}
    {\bf Head of Applied Machine Learning, Clarifai}, June 2015-present. Original data scientist and defined the role of data science within the company - ``improving our machine learning products by designing better training datasets''. Built and deployed convnets to solve computer vision problems for some of our most important verticals as well as our largest customers. Implemented multiple active learning pipelines to automate data cleaning tasks. With company expansion took on management responsibilties and oversaw team growth from scratch to 10 direct reports.   
    
    {\bf Research Staff Member, Howard Hughes Research Laboratories}, June 2012-June 2015. Developed algorithms for mining massive modern datasets and predictive analytics. Conceived of and implemented a system which accurately inferred home locations for over one hundred million Twitter users and combined it with multilingual natural language processing tools to generate forecasts for thousands of international events.


\section{\sc Selected first-author publications}

Ryan Compton, David Jurgens, David Allen, ``Geotagging One Hundred Million Twitter Accounts with Total Variation Minimization'' {\it IEEE BigData 2014} \href{https://arxiv.org/abs/1404.7152}{arXiv:1404.7152} Press coverage: \href{http://www.forbes.com/sites/thomasbrewster/2015/03/07/twitter-location-can-be-determined-through-friends/}{Forbes}, \href{http://www.businessinsider.com/twitter-location-research-at-mentions-cornell-2015-3}{Business Insider}, \href{http://observer.com/2015/03/you-dont-have-to-geotag-your-tweets-to-give-away-your-location/}{New York Observer}, \href{http://dailycaller.com/2015/03/06/how-your-tweets-can-reveal-your-real-location/}{Daily Caller}, \href{http://www.technologyreview.com/view/527246/other-interesting-arxiv-papers-week-ending-may-10-2014/}{MIT Technology Review}, \href{http://www.komando.com/happening-now/299085/forget-gps-hackers-can-pinpoint-your-exact-location-using-social-media/all}{Komando}, \href{http://ryancompton.net/assets/resume/Serene_Risc_Digest_2015_Spring.pdf}{Serene Risc Digest}, \href{http://www.schneier.com/blog/archives/2015/03/geotagging_twit.html}{Schneier on Security}

Ryan Compton, Craig Lee, Tsai-Ching Lu, Lalindra De Silva, Michael Macy, ``Detecting future social unrest in unprocessed Twitter data: Emerging phenomena and big data'', {\it IEEE-ISI}, 2013 {\bf best paper nomination}

Ryan Compton, Stanley Osher and Louis Bouchard, ``Hybrid regularization for MRI reconstruction with static field inhomogeneity correction'', {\it IEEE International Symposium on Biomedical Imaging}, May 2012 (Journal version: Inverse Probl. Imag. 7, 1215-1233 (2013))

\section{\sc Selected talks}

{\it What convnets look at when they look at nudity.} 
\begin{list1}
\item [] Summary: Overview of how deconvnets can be used to visualize representations learned by a convnet trained to filter nsfw photos. Blogged \href{http://blog.clarifai.com/what-convolutional-neural-networks-see-at-when-they-see-nudity/}{here}. Video and deck \href{http://ryancompton.net/2016/12/06/my-talk-at-the-nyc-machine-learning-meetup/}{here}.
\item [] Venues:
\end{list1}

{\it Geotagging One Hundred Million Twitter Accounts with Total Variation Minimization} 
\begin{list1}
\item [] Summary: Techincal presentation of a social network analysis method which can geolocate the overwhelming majority of active Twitter users, independent of their location sharing preferences, using only publicly-visible Twitter data.
\item [] Paper: \href{https://arxiv.org/abs/1404.7152}{arXiv:1404.7152}
\item [] Venues:
\end{list1}

\section{\sc Computer Skills}
C/C++, Python, MATLAB, OpenMP



\end{resume}
\end{document}
